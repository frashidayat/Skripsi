\documentclass[a4paper,twoside]{article}
\usepackage[T1]{fontenc}
\usepackage[bahasa]{babel}
\usepackage{graphicx}
\usepackage{graphics}
\usepackage{float}
\usepackage[cm]{fullpage}
\pagestyle{myheadings}
\usepackage{etoolbox}
\usepackage{setspace} 
\usepackage{lipsum} 
\setlength{\headsep}{30pt}
\usepackage[inner=2cm,outer=2.5cm,top=2.5cm,bottom=2cm]{geometry} %margin
% \pagestyle{empty}

\makeatletter
\renewcommand{\@maketitle} {\begin{center} {\LARGE \textbf{ \textsc{\@title}} \par} \bigskip {\large \textbf{\textsc{\@author}} }\end{center} }
\renewcommand{\thispagestyle}[1]{}
\markright{\textbf{\textsc{AIF401/AIF402 \textemdash Rencana Kerja Skripsi \textemdash Sem. Ganjil 2015/2016}}}

\onehalfspacing
 
\begin{document}

\title{\@judultopik}
\author{\nama \textendash \@npm} 

%tulis nama dan NPM anda di sini:
\newcommand{\nama}{Frasetiawan Hidayat}
\newcommand{\@npm}{2010730121}
\newcommand{\@judultopik}{Analisis Waktu Tempuh Kota Bandung} % Judul/topik anda
\newcommand{\jumpemb}{1} % Jumlah pembimbing, 1 atau 2
\newcommand{\tanggal}{16/02/2017}
\maketitle

\pagenumbering{arabic}

\section{Deskripsi}
Manusia melakukan rutinitas berpindah dari suatu tempat ke tempat lain sebagai contoh: kuliah dan bekerja. Untuk menjalani rutinitas itu, kita melakukan perjalanan melalui jalur yang relatif konstan. Dari jalur yang sering kita lewati itu biasanya terdapat jam-jam tertentu terjadi kemacetan. Dengan demikian kita menginginkan untuk melakukan perjalanan di waktu yang tepat agar waktu tempuh untuk berpindah dari suatu tempat menjadi optimal.

Google Direcction API adalah suatu layanan web yang bisa dimanfaatkan untuk mendapatkan waktu tempuh. Google Direction API adalah sebuah layanan yang disediakan oleh Google untuk menghitung arah alamat statis untuk penempatan konten aplikasi pada peta. Input yang diterima oleh Google Direction API adalah lokasi awal, lokasi akhir dan parameter-parameter opsional lainnya. Untuk bisa saling berkomukasi Google Direction API menggunakan protokol HTTP dengan \textit{port number} 80 dengan output JSON atau XML.

Pada skripsi ini, akan dibuat sebuah perangkat lunak yang dapat menampilkan hasil analisis dari data yang didapatkan dari Google Direction API untuk membantu mengambil keputusan pada jam berapakah harus melakukan perjalanan dengan waktu tempuh yang tercepat. Perangkat lunak ini memanfaatkan salah satu layanan dari Google yaitu Google Direction API. Skripsi ini menggunakan 2 sampel yaitu : menghitung waktu tempuh dari Universitas Katolik Parahyangan dengan alamat Jln. Ciumbuleuit No.94 dan Komplek Amaya Residence, menghitung waktu tempuh dari Universitas Katolik Parahyangan dengan alamat Jln. Ciumbuleuit No.94 dan Komplek Taman Puspa Indah.

\section{Rumusan Masalah}

\begin{itemize}
	\item Bagaimana cara menggunakan Google Direction API dalam bahasa Java?
	\item Bagaimana memanfaatkan layanan Google Direction API untuk memberikan kesimpulan waktu perjalanan terbaik?
	\item Kapan waktu terbaik untuk berangkat/pulang untuk dua sampel tempat yang dimaksud?
\end{itemize}

\section{Tujuan}

\begin{itemize}
	\item memahami menggunakan Google Direction API.
	\item memahami Layanan Google Direction API untuk memberikan kesimpulan waktu perjalanan terbaik.
	\item memutuskan kapan waktu terbaik untuk berangkat/pulang untuk dua sampel yang dimaksud.
\end{itemize}

\section{Deskripsi Perangkat Lunak}

Perangkat lunak akhir yang akan dibuat memiliki fitur minimal sebagai berikut:
\begin{itemize}
	\item Perangkat Lunak dapat menarik data dari Google Direction API.
	\item Perangkat Lunak dapat menghasilkan file output waktu tempuh yang dapat dibuka dengan microsoft excel untuk dibuat grafiknya.
\end{itemize}

\section{Detail Pengerjaan Skripsi}

Bagian-bagian pekerjaan skripsi ini adalah sebagai berikut :
	\begin{enumerate}
		\item Melakukan studi literatur tentang \textit{Google Places API}.
		\item Menganalisis fitur parameter yang dibutuhkan dari layanan \textit{Google Direction API}.
		\item Menganalisis output dari permintaan layanan.
		\item Merancang perangkat lunak hasil analisis waktu tempuh.
		\item Mengimplementasikan perangkat lunak.
		\item Melakukan pengujian dan eksperimen sesuai dengan sample.
		\item Menganalisa hasil keluaran program.
		\item Menulis dokumen skripsi.
	\end{enumerate}

\section{Rencana Kerja}

Berikut adalah rencana kerja yang akan dilakukan pada saat menagmbil kuliah AIF402 Skripsi 2 untuk menyelesaikan skripsi ini :

\begin{center}
  \begin{tabular}{ | c | c | c | c | l |}
    \hline
    1*  & 2*(\%) & 3*(\%) & 4*(\%) &5*\\ \hline \hline
    1   & 15  &    & 15  & \\ \hline
    2   & 15  &    & 15  & \\ \hline
    3   & 10  &    & 10  & \\ \hline
    4   & 15  &    &  15 & \\ \hline
    5   & 15  &    &  15 & \\ \hline
    6   & 10  &    &  10 & \\ \hline
    7   & 10 &    & 10 & \\ \hline
    8  & 10 &    & 10 & \\ \hline
    Total  & 100  &    & 100 &  \\ \hline
                          \end{tabular}
\end{center}

Keterangan (*)\\
1 : Bagian pengerjaan Skripsi (nomor disesuaikan dengan detail pengerjaan di bagian 5)\\
2 : Persentase total \\
3 : Persentase yang akan diselesaikan di Skripsi 1 \\
4 : Persentase yang akan diselesaikan di Skripsi 2 \\
5 : Penjelasan singkat apa yang dilakukan di S1 (Skripsi 1) atau S2 (skripsi 2)

\newpage 
\vspace{1 cm}
\centering Bandung, \tanggal\\
\vspace{2cm} \nama \\ 
\vspace{1cm}

Menyetujui, \\
\ifdefstring{\jumpemb}{2}{
\vspace{1.5cm}
\begin{centering} Menyetujui,\\ \end{centering} \vspace{0.75cm}
\begin{minipage}[b]{0.45\linewidth}
% \centering Bandung, \makebox[0.5cm]{\hrulefill}/\makebox[0.5cm]{\hrulefill}/2013 \\
\vspace{2cm} Nama: \makebox[3cm]{\hrulefill}\\ Pembimbing Utama
\end{minipage} \hspace{0.5cm}
\begin{minipage}[b]{0.45\linewidth}
% \centering Bandung, \makebox[0.5cm]{\hrulefill}/\makebox[0.5cm]{\hrulefill}/2013\\
\vspace{2cm} Nama: \makebox[3cm]{\hrulefill}\\ Pembimbing Pendamping
\end{minipage}
\vspace{0.5cm}
}{
% \centering Bandung, \makebox[0.5cm]{\hrulefill}/\makebox[0.5cm]{\hrulefill}/2013\\
\vspace{2cm} Nama: Pascal Alfadian Nugroho, S.Kom, M.Kom\\ Pembimbing Tunggal
}

\end{document}

