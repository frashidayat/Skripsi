\documentclass[a4paper,twoside]{article}
\usepackage[T1]{fontenc}
\usepackage[bahasa]{babel}
\usepackage{graphicx}
\usepackage{graphics}
\usepackage{float}
\usepackage[cm]{fullpage}
\pagestyle{myheadings}
\usepackage{etoolbox}
\usepackage{setspace} 
\usepackage{lipsum} 
\setlength{\headsep}{30pt}
\usepackage[inner=2cm,outer=2.5cm,top=2.5cm,bottom=2cm]{geometry} %margin
% \pagestyle{empty}

\makeatletter
\renewcommand{\@maketitle} {\begin{center} {\LARGE \textbf{ \textsc{\@title}} \par} \bigskip {\large \textbf{\textsc{\@author}} }\end{center} }
\renewcommand{\thispagestyle}[1]{}
\markright{\textbf{\textsc{AIF401/AIF402 \textemdash Rencana Kerja Skripsi \textemdash Sem. Ganjil 2015/2016}}}

\onehalfspacing
 
\begin{document}

\title{\@judultopik}
\author{\nama \textendash \@npm} 

%tulis nama dan NPM anda di sini:
\newcommand{\nama}{Frasetiawan Hidayat}
\newcommand{\@npm}{2010730121}
\newcommand{\@judultopik}{Analisis Waktu Tempuh Kota Bandung} % Judul/topik anda
\newcommand{\jumpemb}{1} % Jumlah pembimbing, 1 atau 2
\newcommand{\tanggal}{16/01/2017}
\maketitle

\pagenumbering{arabic}

\section{Deskripsi}
Waktu adalah salah satu elemen yang paling penting dalam kehidupan manusia. Sebagai manusia, pasti ingin mengefektifkan waktu agar semua rencana yang sudah dibuat berjalan dengan lancar. Salah satu permasalahan yang kini terjadi adalah kemacetan yang menyita waktu dan membuat kita sering terlambat untuk melakukan rutinitas.

Kemacetan adalah suatu masalah yang sudah lama terjadi di kota-kota besar. Kemacetan juga menghambat aktifitas manusia untuk bermobilitas tinggi dan merusak rencana yang dibuat sedemikian rupa agar dapat terlaksanakan dengan baik dan tepat waktu. Bandung adalah salah satu kota besar yang memiliki permasalahan kemacetan. 

Kita sebagai manusia melakukan rutinitas berpindah dari suatu tempat ke tempat lain sebagai contoh : kuliah, bekerja. Untuk menjalani rutinitas itu, kita melakukan perjalanan melalui jalur yang sering kita lewati. Dari jalur yang sering kita lewati itu biasanya terdapat jam-jam tertentu terjadi kemacetan. Dengan demikian kita menginginkan untuk melakukan perjalanan di waktu yang tepat agar waktu tempuh untuk berpindah dari suatu tempat menjadi optimal.

Pada skripsi ini, akan dibuat sebuah perangkat lunak yang dapat menampilkan analisis berbentuk bagan untuk membantu mengambil keputusan pada jam berapakah harus melakukan perjalanan dengan waktu tempuh yang tercepat. Perangkat lunak ini memanfaatkan salah satu teknologi Google yaitu Google Direction. Google Direction adalah sebuah layanan yang disediakan oleh Google menggunakan protokol HTTP untuk menghitung arah alamat statis untuk penempatan konten aplikasi pada peta. Input yang diterima oleh Google Direction adalah lokasi awal dan lokasi akhir.

\section{Rumusan Masalah}

\begin{itemize}
	\item Bagaimana protokol HTTP?
	\item Bagaimana komunikasi layanan Google Direction?
	\item Bagaimana mengimplementasikan komunikasi Google Direction di Java?
	\item Bagaimana mengimplementasikan komunikasi Google Direction dengan permintaan beberapa waktu?
	\item Bagaimana menganalisis hasil dari permintaan komunikasi Google Direction pada suatu jalur?
\end{itemize}

\section{Tujuan}

\begin{itemize}
	\item memahami protokol HTTP.
	\item memahami layanan web dari Google Direction.
	\item mengimplementasikan komunikasi layanan web Google Direction pada Java.
	\item menganalisis waktu tempuh terbaik pada suatu jalur.
\end{itemize}

\section{Deskripsi Perangkat Lunak}

Perangkat lunak akhir yang akan dibuat memiliki fitur minimal sebagai berikut:
\begin{itemize}
	\item Pengguna dapat menentukan titik awal dan titik tujuan untuk dihitung waktu tempuh optimalnya.
	\item Pengguna dapat melihat hasil analisis waktu tempuh optimal berupa bagan.
	\item Pengguna dapat mengatur hasil analisis perhari atau perminggu.
\end{itemize}

\section{Detail Pengerjaan Skripsi}

Bagian-bagian pekerjaan skripsi ini adalah sebagai berikut :
	\begin{enumerate}
		\item Melakukan studi literatur tentang protokol HTTP.
		\item Melakukan studi literatur tentang \textit{Web Service}.
		\item Melakukan studi literatur tentang \textit{Google Places API} dan \textit{Google Direction API}.
		\item Menganalisis protokol HTTP.
		\item Menganalisis \textit{Web Service}.
		\item Mempelajari dan menganalisis fitur dari layanan \textit{Google Places API} dan \textit{Google Direction API}.
		\item Merancang perangkat lunak hasil analisis waktu tempuh.
		\item Mengimplementasikan komunikasi layanan \textit{Google Direction}.
		\item Melakukan pengujian dan eksperimen sesuai dengan sample.
		\item Menulis dokumen skripsi.
	\end{enumerate}

\section{Rencana Kerja}

Berikut adalah rencana kerja yang akan dilakukan pada saat menagmbil kuliah AIF402 Skripsi 2 untuk menyelesaikan skripsi ini :

\begin{center}
  \begin{tabular}{ | c | c | c | c | l |}
    \hline
    1*  & 2*(\%) & 3*(\%) & 4*(\%) &5*\\ \hline \hline
    1   & 5  &    & 5  & \\ \hline
    2   & 5  &    & 5  & \\ \hline
    3   & 5  &    & 5  & \\ \hline
    4   & 7  &    & 7  & \\ \hline
    5   & 7  &    & 7  & \\ \hline
    6   & 7  &    & 7  & \\ \hline
    7   & 18 &    & 18 & \\ \hline
    8   & 18 &    & 18 & \\ \hline
    9   & 18 &    & 18 & \\ \hline
    10  & 10 &    & 10 & \\ \hline
    Total  & 100  &    & 100 &  \\ \hline
                          \end{tabular}
\end{center}

Keterangan (*)\\
1 : Bagian pengerjaan Skripsi (nomor disesuaikan dengan detail pengerjaan di bagian 5)\\
2 : Persentase total \\
3 : Persentase yang akan diselesaikan di Skripsi 1 \\
4 : Persentase yang akan diselesaikan di Skripsi 2 \\
5 : Penjelasan singkat apa yang dilakukan di S1 (Skripsi 1) atau S2 (skripsi 2)
\newpage
\vspace{1cm}
\centering Bandung, \tanggal\\
\vspace{2cm} \nama \\ 
\vspace{1cm}

Menyetujui, \\
\ifdefstring{\jumpemb}{2}{
\vspace{1.5cm}
\begin{centering} Menyetujui,\\ \end{centering} \vspace{0.75cm}
\begin{minipage}[b]{0.45\linewidth}
% \centering Bandung, \makebox[0.5cm]{\hrulefill}/\makebox[0.5cm]{\hrulefill}/2013 \\
\vspace{2cm} Nama: \makebox[3cm]{\hrulefill}\\ Pembimbing Utama
\end{minipage} \hspace{0.5cm}
\begin{minipage}[b]{0.45\linewidth}
% \centering Bandung, \makebox[0.5cm]{\hrulefill}/\makebox[0.5cm]{\hrulefill}/2013\\
\vspace{2cm} Nama: \makebox[3cm]{\hrulefill}\\ Pembimbing Pendamping
\end{minipage}
\vspace{0.5cm}
}{
% \centering Bandung, \makebox[0.5cm]{\hrulefill}/\makebox[0.5cm]{\hrulefill}/2013\\
\vspace{2cm} Nama: \makebox[3cm]{\hrulefill}\\ Pembimbing Tunggal
}

\end{document}

