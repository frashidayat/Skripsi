\chapter{Kesimpulan dan Saran}
\label{chap:kesimpulandansaran}

\section{Kesimpulan}
\label{sec:kesimpulan}

Dari hasil pembangunan perangkat lunak analisis waktu tempuh kota Bandung dengan memanfaatkan layanan Google Direction dan hasil data yang didapatkan, didapatkanlah kesimpulan-kesimpulan sebagai berikut :

\begin{enumerate} 
	\item Telah berhasil mengimplementasi Google Direction API dalam bahasa pemrograman Java menggunakan \textit{library} jsoup. Dengan \textit{library} jsoup, dapat melakukan \textit{request} ke layanan Google Direction menggunakan url yang telah mengandung parameter-parameter yang diperlukan. 
	\item Telah berhasil mengekstraksi data dengan menggunakan \textit{library} JSON. Dengan \textit{library} JSON, \textit{response} dari \textit{request} ke layanan Google Direction dapat diekstraksi dengan cara mengekstrak nilai dari \textit{duration\_in\_traffic}.
	\item Sesuai dengan apa yang telah dipaparkan pada subbab \ref{subsec:pengujianeksperimental}, Waktu terbaik untuk melakukan perjalanan pada sampel 1 adalah \textbf{hari rabu pukul 4}. Pada sampel 2 waktu terbaik untuk melakukan perjalanan adalah hari \textbf{hari rabu pukul 4}.
\end{enumerate}

\section{Saran}
\label{sec:saran}

Dari hasil penelitian termasuk kesimpulan yang didapat, berikut adalah beberapa saran untuk pengembangan:

\begin{enumerate} 
	\item Penelitian ini memanfaatkan Google Direction API dengan mengembalikan \textit{response} dengan format JSON. Oleh karena itu, sebaiknya perangkat lunak yang akan dibangun bisa menangani segala format keluaran/\textit{response} dari layanan ini.
	\item Pada parameter \textit{APIKEY} yang terintergrasi dengan akun \textit{Google} dimana parameter tersebut digunakan untuk melakukan \textit{request} ke layanan Google Direction memiliki batas untuk melakukan \textit{request}. Oleh karena itu, sebaiknya perhatikan kuota sebelum pengembang melakukan \textit{request} atau memperbesar kuota \textit{request} dengan membayar sejumlah uang ke pihak Google untuk berjaga-jaga terjadi melebihi batas kuota pada saat melakukan \textit{request}.
\end{enumerate}