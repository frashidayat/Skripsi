%versi 2 (8-10-2016)
\chapter{Landasan Teori}
\label{chap:teori}

Pada bab ini akan diuraikan teori-teori yang akan digunakan untuk pembangunan aplikasi ke analisis kota Bandung. Teori-teori tersebut adalah penjelasan tentang Google sebagai web service, Penjelasan tentang \textit{Google API}, Penjelasan tentang \textit{Google Direction API}, Penjelasan tentang protokol HTTP dan teori JSON.

\section{Google}
\label{sec:google}

\subsection{Google Web Service}
\label{subsec:webservicegoogle}

\subsection{Google API}
\label{subsec:googleapi}

\subsection{Google Direction API}
\label{subsec:googledirapi}

\section{Protokol HTTP}
\label{sec:prothttp} 


\section{Format JSON}
\label{sec:json}

JSON (JavaScript Object Notation) adalah format pertukaran data yang ringan, mudah dibaca dan ditulis oleh manusia, serta mudah diterjemahkan dan dibuat (generate) oleh komputer. Format ini dibuat berdasarkan bagian dari Bahasa Pemprograman JavaScript, Standar ECMA-262 Edisi ke-3 - Desember 1999. JSON merupakan format teks yang tidak bergantung pada bahasa pemprograman apapun karena menggunakan gaya bahasa yang umum digunakan oleh programmer keluarga C termasuk C, C++, C#, Java, JavaScript, Perl, Python dll. Oleh karena sifat-sifat tersebut, menjadikan JSON ideal sebagai bahasa pertukaran-data.

\subsection{Struktur JSON}
\label{subsec:stukturjson}
JSON terbuat dari dua struktur :
\begin{itemize}
	\item Kumpulan pasangan nama/nilai.
	\item Daftar nilai terurutkan (an ordered list of values).
\end{itemize} 

Struktur-struktur data ini disebut sebagai struktur data universal. Pada dasarnya, semua bahasa pemprograman moderen mendukung struktur data ini dalam bentuk yang sama maupun berlainan. Hal ini pantas disebut demikian karena format data mudah dipertukarkan dengan bahasa-bahasa pemprograman yang juga berdasarkan pada struktur data ini.
 
