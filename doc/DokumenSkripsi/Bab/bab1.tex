%versi 2 (8-10-2016) 
\chapter{Pendahuluan}
\label{chap:intro}
   
\section{Latar Belakang}
\label{sec:label}
Waktu adalah elemen penting dalam hidup kita. Kita sebagai manusia merencanakan semua kegiatan dan rutinitas yang akan dilakukan agar kegiatan dan rutinitas tersebut bisa berjalan dengan baik dan sebagai mana mestinya. Waktu adalah kunci dari segala rencana yang akan dibuat.

Terkadang di kota-kota besar kemacetan akan menjadi sangat merepotkan dan akan menghancurkan semua rencana yang dibuat. Efeknya adalah keterlambatan
yang akan mempengaruhi seluruh rangkaian kegiatan yang telah direncanakan. Bandung adalah salah satunya dari kota besar yang memiliki permasalahan kemacetan ini dan terkadang kemacetan sendiri tidak dapat diprediksi dan akan sangat merepotkan.

Dengan demikian, untuk merencanakan segalanya agar berjalan sesuai dengan rencana, perlu untuk mengoptimalkan waktu tempuh dari jalur yang sering kita gunakan agar tidak terjebak dalam kemacetan yang akan merepotkan. Kemacetan ini sendiri bisa dianalisis dengan menentukan pada pukul berapa sajakah terjadi kemacetan pada jalur yang kita tempuh.

Salah satu teknologi yang telah ada, \textit{Google Direction} adalah suatu layanan web untuk menghitung arah antar lokasi. Layanan web ini didesain menghitung arah alamat statis untuk penempatan konten aplikasi pada peta (\textit{Google Maps}). Dengan layanan web ini juga kita bisa mendapatkan data: waktu tempuh dari lokasi awal sampai lokasi tujuan, jalan yang ditempuh.

Layanan web sendiri adalah setiap layanan yang tersedia melalui internet.
Layanan web ini sendiri menggunakan suatu format sistem pesan yang terstandarisasi yang bisa diakses oleh aplikasi lain. Layanan web ini juga tidak terikat pada satu system operasi atau Bahasa pemrograman agar bisa diakses oleh aplikasi lain. contoh format dari layanan web adalah JSON dan XML.

Google Direction sendiri menggunakan protokol HTTP untuk bisa saling berkomunikasi dengan aplikasi. Protokol HTTP merupakan protokol yang berjalan diatas protokol TCP pada port 80 yang digunakan untuk mengirim dokumen atau halaman. Pesan protokol http diformat untuk dapat ditampilkan pada aplikasi.

Dalam penelitian ini, akan membangun aplikasi untuk mengeluarkan hasil analisis waktu tempuh optimal dengan memanfaatkan teknologi dari \textit{google} yaitu \textit{google direction}.

\section{Rumusan Masalah}
\label{sec:rumusan}
Berdasarkan latar belakang masalah yang telah dijelaskan, rumusan masalah pada penelitian ini adalah:
\begin{itemize}
	\item Bagaimana protokol HTTP?
	\item Bagaimana komunikasi layanan Google Direction?
	\item Bagaimana mengimplementasikan komunikasi Google Direction di Java?
	\item Bagaimana mengimplementasikan komunikasi Google Direction dengan permintaan beberapa waktu?
	\item Bagaimana menganalisis hasil dari permintaan komunikasi Google Direction pada suatu jalur?
\end{itemize}

\section{Tujuan}
\label{sec:tujuan}
Berdasarkan rumusan masalah di atas, maka tujuan dari penelitian ini adalah:
\begin{itemize}
	\item memahami protokol HTTP.
	\item memahami layanan web dari Google Direction.
	\item mengimplementasikan komunikasi layanan web Google Direction pada Java.
	\item menganalisis waktu tempuh terbaik pada suatu jalur.
\end{itemize}

\section{Batasan Masalah}
\label{sec:batasan}
Batasan masalah yang akan digunakan untuk peneliatian ini adalah:
\begin{enumerate}
	\item Penetapan tujuan awal dari program yang dibuat adalah alamat dari Universitas Katolik Parahyangan.
	\item Penetapan tujuan akhir dari program yang dibuat adalah alamat dari rumah pembimbing penulis dan alamat rumah penulis.
	\item Waktu tempuh dihitung setiap jam dalam satu hari.
	\item Waktu tempuh dihitung setiap hari dalam seminggu.
\end{enumerate}

\section{Metodologi}
\label{sec:metlit}
Dalam penyusunan skripsi ini mengikuti langkah-langkah metodologi penelitian sebagai berikut :
\begin{enumerate}
	\item Melakukan studi pustaka untuk dijadikan referensi dalam melakukan pembangunan aplikasi Analisis waktu tempuh kota Bandung,
	\item Melakukan analisis protokol HTTP, layanan web untuk mendapatkan hasil waktu tempuh dari tujuan asal ke tujuan akhir,
	\item Melakukan perancangan perangkat lunak.
	\item Melakukan uji coba sesuai dengan sample,
	\item Melakukan penarikan kesimpulan dan saran pada hasil analisis tersebut.
\end{enumerate}


\section{Sistematika Pembahasan}
\label{sec:sispem}
Sistematika penulisan laporan pada skripsi ini adalah sebagai berikut :
\begin{enumerate}
	\item Bab Pendahuluan\\
	Bab 1 berisi latar belakang, rumusan masalah, tujuan, batasan masalah, metodologi penelitian, dan sistematika pembahasan dalam pelaksanaan penelitian ini.
	\item Bab Dasar Teori\\
	Bab 2 berisi tentang definisi-definisi dasar teori tentang protokol HTTP, layanan web.
	\item Bab Analisis\\
	Bab 3 berisi analisis protokol HTTP, layanan web dan analisis perangkat lunak.
	\item Bab Perancangan\\
	Bab 4 berisi tentang pembahasan menegenai perancangan perangkat lunak untuk menampilkan hasil analisis waktu tempuh.
	\item Bab Impelemntasi dan Pengujian\\
	Bab 5 berisi tentang pengimplementasian dari komunikasi antara aplikasi dan layanan web dan bagan hasil analisis waktu tempuhnya.
	\item Bab Kesimpulan dan Saran\\
	Bab 6 berisi penarikan kesimpulan selama menyelesaikan skripsi dan saran yang diusulkan untuk penelitian berikutnya.
\end{enumerate}