%versi 2 (8-10-2016) 
\chapter{Pendahuluan}
\label{chap:intro}
   
\section{Latar Belakang}
\label{sec:label}
\textsc{
Bagian ini akan diisi dengan apa yang melatarbelakangi pembuatan template skripsi ini.
Termasuk juga masalah-masalah yang akan dihadapi untuk membuatnya, termasuk kurangnya kemampuan penguasaan \LaTeX{} sehingga template ini dibuat dengan mengandalkan berbagai contoh yang tersebar di dunia maya, yang digabung-gabung menjadi satu jua.
Bagian lain juga akan dilengkapi, untuk sementara diisi dengan lorem ipsum versi bahasa inggris.
}

\kant[5-10]

\section{Rumusan Masalah}
\label{sec:rumusan}
Berdasarkan latar belakang masalah yang telah dijelaskan, rumusan masalah pada penelitian ini adalah:
\begin{itemize}
	\item Bagaimana protokol HTTP?
	\item Bagaimana komunikasi layanan Google Direction?
	\item Bagaimana mengimplementasikan komunikasi Google Direction di Java?
	\item Bagaimana mengimplementasikan komunikasi Google Direction dengan permintaan beberapa waktu?
	\item Bagaimana menganalisis hasil dari permintaan komunikasi Google Direction pada suatu jalur?
\end{itemize}

\section{Tujuan}
\label{sec:tujuan}
Berdasarkan rumusan masalah di atas, maka tujuan dari penelitian ini adalah:
\begin{itemize}
	\item memahami protokol HTTP.
	\item memahami layanan web dari Google Direction.
	\item mengimplementasikan komunikasi layanan web Google Direction pada Java.
	\item menganalisis waktu tempuh terbaik pada suatu jalur.
\end{itemize}

\section{Batasan Masalah}
\label{sec:batasan}
Batasan masalah yang akan digunakan untuk peneliatian ini adalah:
\begin{enumerate}
	\item Penetapan tujuan awal dari program yang dibuat adalah alamat dari Universitas Katolik Parahyangan.
	\item Penetapan tujuan akhir dari program yang dibuat adalah alamat dari rumah pembimbing penulis dan alamat rumah penulis.
	\item Waktu tempuh dihitung setiap jam dalam satu hari.
	\item Waktu tempuh dihitung setiap hari dalam seminggu.
\end{enumerate}

\section{Metodologi}
\label{sec:metlit}
Dalam penyusunan skripsi ini mengikuti langkah-langkah metodologi penelitian sebagai berikut :
\begin{enumerate}
	\item Melakukan studi pustaka untuk dijadikan referensi dalam melakukan pembangunan aplikasi Analisis waktu tempuh kota Bandung,
	\item Melakukan analisis protokol HTTP, layanan web untuk mendapatkan hasil waktu tempuh dari tujuan asal ke tujuan akhir,
	\item Melakukan perancangan perangkat lunak.
	\item Melakukan uji coba sesuai dengan sample,
	\item Melakukan penarikan kesimpulan dan saran pada hasil analisis tersebut.
\end{enumerate}


\section{Sistematika Pembahasan}
\label{sec:sispem}
Sistematika penulisan laporan pada skripsi ini adalah sebagai berikut :
\begin{enumerate}
	\item Bab Pendahuluan\\
	Bab 1 berisi latar belakang, rumusan masalah, tujuan, batasan masalah, metodologi penelitian, dan sistematika pembahasan dalam pelaksanaan penelitian ini.
	\item Bab Dasar Teori\\
	Bab 2 berisi tentang definisi-definisi dasar teori tentang protokol HTTP, layanan web.
	\item Bab Analisis\\
	Bab 3 berisi analisis protokol HTTP, layanan web dan analisis perangkat lunak.
	\item Bab Perancangan\\
	Bab 4 berisi tentang pembahasan menegenai perancangan perangkat lunak untuk menampilkan hasil analisis waktu tempuh.
	\item Bab Impelemntasi dan Pengujian\\
	Bab 5 berisi tentang pengimplementasian dari komunikasi antara aplikasi dan layanan web dan bagan hasil analisis waktu tempuhnya.
	\item Bab Kesimpulan dan Saran\\
	Bab 6 berisi penarikan kesimpulan selama menyelesaikan skripsi dan saran yang diusulkan untuk penelitian berikutnya.
\end{enumerate}