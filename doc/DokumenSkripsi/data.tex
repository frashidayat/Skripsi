%_____________________________________________________________________________
%=============================================================================
% data.tex v8 (02-10-2016) \ldots dibuat oleh Lionov - Informatika FTIS UNPAR
%
% Perubahan pada versi 8 (02-10-2016)
%	- Perubahan keterangan pada spacing: Otomatis spasi 1 untuk buku skripsi 
%	  final dan 1.5 untuk buku sidang
%	- Penggunaan kantlipsum
%_____________________________________________________________________________
%=============================================================================

%=============================================================================
% 								PETUNJUK
%=============================================================================
% Ini adalah file data (data.tex)
% Masukkan ke dalam file ini, data-data yang diperlukan oleh template ini
% Cara memasukkan data dijelaskan di setiap bagian
% Data yang WAJIB dan HARUS diisi dengan baik dan benar adalah SELURUHNYA !!
% Hilangkan tanda << dan >> jika anda menemukannya
%=============================================================================

%_____________________________________________________________________________
%=============================================================================
% 								BAGIAN 0
%=============================================================================
% PERHATIAN!! PERHATIAN!! Bagian ini hanya ada untuk sementara saja
% Jika "DAFTAR ISI" tidak bisa berada di bagian tengah halaman, isi dengan XXX
% jika sudah benar posisinya, biarkan kosong (i.e. \daftarIsiError{ })
%=============================================================================
\daftarIsiError{ }
%=============================================================================

%_____________________________________________________________________________
%=============================================================================
% 								BAGIAN I
%=============================================================================
% Tambahkan package2 lain yang anda butuhkan di sini
%=============================================================================
\usepackage{booktabs} 
\usepackage[table]{xcolor}
\usepackage{longtable}
\usepackage{amssymb}
\usepackage{todo}
\usepackage{verbatim} 		%multilne comment
\usepackage{pgfplots}
\usepackage{listings}
%=============================================================================

%_____________________________________________________________________________
%=============================================================================
% 								BAGIAN II
%=============================================================================
% Mode dokumen: menetukan halaman depan dari dokumen, apakah harus mengandung 
% prakata/pernyataan/abstrak dll (termasuk daftar gambar/tabel/isi) ?
% - kosong : tidak ada halaman depan sama sekali (untuk dokumen yang 
%            dipergunakan pada proses bimbingan)
% - cover : cover saja tanpa daftar isi, gambar dan tabel
% - sidang : cover, daftar isi, gambar, tabel 
% - sidang_akhir : mode sidang + abstrak + abstract
% - final : seluruh halaman awal dokumen (untuk cetak final)
% Jika tidak ingin mencetak daftar tabel/gambar (misalkan karena tidak ada 
% isinya), edit manual di baris 439 dan 440 pada file main.tex
%=============================================================================
% \mode{kosong}
% \mode{cover}
% \mode{sidang}
%\mode{sidang_akhir}
\mode{final} 
%=============================================================================

%_____________________________________________________________________________
%=============================================================================
% 								BAGIAN III
%=============================================================================
% Line numbering: penomoran setiap baris, otomatis di-reset setiap berganti
% halaman
% - yes: setiap baris diberi nomor
% - no : baris tidak diberi nomor, otomatis untuk mode final
%=============================================================================
\linenumber{}
%=============================================================================

%_____________________________________________________________________________
%=============================================================================
% 								BAGIAN IV
%=============================================================================
% Linespacing: jarak antara baris 
% - single	: wajib (dan otomatis jika ingin mencetak buku skripsi, opsi yang 
%			  disediakan untuk bimbingan, jika pembimbing tidak keberatan 
%			  (untuk menghemat kertas)
% - onehalf	: default dan wajib (dan otomatis) jika ingin mencetak dokumen
%             untuk sidang.
% - double 	: jarak yang lebih lebar lagi, jika pembimbing berniat memberi 
%             catatan yg banyak di antara baris (dianjurkan untuk bimbingan)
%=============================================================================
\linespacing{single}
%\linespacing{onehalf}
%\linespacing{double}
%=============================================================================

%_____________________________________________________________________________
%=============================================================================
% 								BAGIAN V
%=============================================================================
% Tidak semua skripsi memuat gambar dan/atau tabel. Untuk skripsi yang seperti
% itu, tidak diperlukan Daftar Gambar dan Daftar Tabel. Sayangnya hal ini 
% sulit dilakukan secara manual karena membutuhkan kedisiplinan pengguna 
% template.  
% Jika tidak akan menampilkan Daftar Gambar/Tabel, isi dengan NO. Jika ingin
% menampilkan, kosongkan parameter (i.e. \gambar{ }, \tabel{ })
%=============================================================================
\gambar{ }
\tabel{ }
%=============================================================================

%_____________________________________________________________________________
%=============================================================================
% 								BAGIAN VI
%=============================================================================
% Bab yang akan dicetak: isi dengan angka 1,2,3 s.d 9, sehingga bisa digunakan
% untuk mencetak hanya 1 atau beberapa bab saja
% Jika lebih dari 1 bab, pisahkan dengan ',', bab akan dicetak terurut sesuai 
% urutan bab (e.g. \bab{1,2,3}).
% Untuk mencetak seluruh bab, kosongkan parameter (i.e. \bab{ })  
% Catatan: Jika ingin menambahkan bab ke-10 dan seterusnya, harus dilakukan 
% secara manual
%=============================================================================
\bab{}
%=============================================================================

%_____________________________________________________________________________
%=============================================================================
% 								BAGIAN VII
%=============================================================================
% Lampiran yang akan dicetak: isi dengan huruf A,B,C s.d I, sehingga bisa 
% digunakan untuk mencetak hanya 1 atau beberapa lampiran saja
% Jika lebih dari 1 lampiran, pisahkan dengan ',', lampiran akan dicetak 
% terurut sesuai urutan lampiran (e.g. \bab{A,B,C}).
% Jika tidak ingin mencetak lampiran apapun, isi dengan -1 (i.e. \lampiran{-1})
% Untuk mencetak seluruh mapiran, kosongkan parameter (i.e. \lampiran{ })  
% Catatan: Jika ingin menambahkan lampiran ke-J dan seterusnya, harus 
% dilakukan secara manual
%=============================================================================
\lampiran{ }
%=============================================================================

%_____________________________________________________________________________
%=============================================================================
% 								BAGIAN VIII
%=============================================================================
% Data diri dan skripsi/tugas akhir
% - namanpm: Nama dan NPM anda, penggunaan huruf besar untuk nama harus benar
%			 dan gunakan 10 digit npm UNPAR, PASTIKAN BAHWA BENAR !!!
%			 (e.g. \namanpm{Jane Doe}{1992710001}
% - judul : Dalam bahasa Indonesia, perhatikan penggunaan huruf besar, judul
%			tidak menggunakan huruf besar seluruhnya !!! 
% - tanggal : isi dengan {tangga}{bulan}{tahun} dalam angka numerik, jangan 
%			  menuliskan kata (e.g. AGUSTUS) dalam isian bulan
%			  Tanggal ini adalah tanggal dimana anda akan melaksanakan sidang 
%			  ujian akhir skripsi/tugas akhir
% - pembimbing: isi dengan pembimbing anda, lihat daftar dosen di file dosen.tex
%				jika pembimbing hanya 1, kosongkan parameter kedua 
%				(e.g. \pembimbing{\JND}{  } ) , \JND adalah kode dosen
% - penguji : isi dengan para penguji anda, lihat daftar dosen di file dosen.tex
%				(e.g. \penguji{\JHD}{\JCD} ) , \JND dan \JCD adalah kode dosen
% !!Lihat singkatan pembimbing dan penguji anda di file dosen.tex
%=============================================================================
\namanpm{Frasetiawan Hidayat}{2010730121}	%hilangkan tanda << & >>
\tanggal{31}{7}{2017}			%hilangkan tanda << & >>
\pembimbing{\LCA}{} %hilangkan tanda << & >>    
\penguji{\CHW}{\HUH} 				%hilangkan tanda << & >>
%=============================================================================

%_____________________________________________________________________________
%=============================================================================
% 								BAGIAN IX
%=============================================================================
% Judul dan title : judul bhs indonesia dan inggris
% - judulINA: judul dalam bahasa indonesia
% - judulENG: title in english
% PERHATIAN: - langsung mulai setelah '{' awal, jangan mulai menulis di baris 
%			   bawahnya
%			 - Gunakan \texorpdfstring{\\}{} untuk pindah ke baris baru
%			 - Judul TIDAK ditulis dengan menggunakan huruf besar seluruhnya !!
%			 - Gunakan perintah \texorpdfstring{\\}{} untuk baris baru
%=============================================================================
\judulINA{Analisis Waktu Tempuh Kota Bandung (Studi Kasus : Antara UNPAR dengan Komplek Amaya Residence dan Jalan Puspa Utara)}
\judulENG{Analysis of Travel Time Bandung city (Case Study: Between UNPAR with Amaya Residence and Jalan Puspa Utara)}
%_____________________________________________________________________________
%=============================================================================
% 								BAGIAN X
%=============================================================================
% Abstrak dan abstract : abstrak bhs indonesia dan inggris
% - abstrakINA: abstrak bahasa indonesia
% - abstrakENG: abstract in english
% PERHATIAN: langsung mulai setelah '{' awal, jangan mulai menulis di baris 
%			 bawahnya
%=============================================================================
\abstrakINA{Dalam melakukan suatu perjalanan , manusia melalui suatu jalur yang relatif konstan dimana jalur tersebut akan menjadi rutinitas yang akan dilalui. Dari jalur tersebut sering kali terjadi kemacetan dan biasanya kemacetan itu terjadi pada jam-jam tertentu.

Pada kota-kota besar sering terjadi kemacetan. Efeknya adalah keterlambatan yang akan mempengaruhi seluruh rangkaian kegiatan yang telah direncanakan. Bandung adalah salah satunya dari kota besar yang sering mengalami kemacetan ini dan terkadang kemacetan sendiri tidak dapat diprediksi.Kemacetan ini sendiri bisa dianalisis dengan menentukan pada pukul berapa sajakah terjadi kemacetan pada jalur yang ditempuh.

Dengan memanfaatkan Google Direction yang dimana Google Direction itu sendiri adalah suatu layanan web untuk menghitung arah antar lokasi. Dengan layanan web ini, pengguna bisa mendapatkan data waktu tempuh dari lokasi awal sampai lokasi tujuan. cara mendapatkan data waktu tempuh tersebut adalah dengan input berupa URL beserta dengan parameter wajib dan beberapa parameter opsional. Parameter wajib yang dimasukan kedalam URL adalah \textit{origin} yang berupa suatu titik \textit{longitude} dan \textit{latitude} dari tempat asal keberangkatan, \textit{destination} yang berupa suatu titik \textit{longitude} dan \textit{latitude} dari tempat tujuan, dan \textit{keys} yang didapatkan dari Google Console. Pengguna menyematkan parameter-parameter tersebut kedalam URL dan akan menghasilkan suatu \textit{output} dengan menggunakan suatu format. Salah satu dari format itu adalah format JSON.

Aplikasi sederhana yang akan dibangun bertujuan untuk mengekstraksi data waktu tempuh dari input request dalam satu hari selama satu minggu. Aplikasi tersebut berbasis Java dengan memanfaatkan \textit{library} jsoup untuk bisa melakukan \textit{request} ke layanan Google Direction dan \textit{library} JSON untuk melakukan ekstraksi data waktu tempuh. Pengujian dari aplikasi sederhana ini dilakukan dengan menggunakan \textit{test case} dengan melakukan permintaan pada suatu hari. Berdasarkan hasil pengujian, aplikasi dapat berjalan dengan baik dan memberikan keluaran file .csv yang akan dianalisis untuk memberikan waktu terbaik dalam melakukan perjalanan dengan bantuan aplikasi Microsoft Excel. Hasil pengujian aplikasi sederhana ini membuktikan bahwa Google Direction API dapat dimanfaatkan untuk menganalisis waktu tempuh antar titik agar mendapatkan waktu tempuh yang optimal.}

\abstrakENG{In doing a travel, man through a relatively constant path where the path will be a routine to be traversed. From this path there is often a traffic jam and usually the jam occurs at certain hours.

In big cities there are frequent congestion. The effect is the delay that will affect the whole set of planned activities. Bandung is one of the big cities that often experience this bottleneck and sometimes congestion itself can not be predicted. The congestion itself can be analyzed by determining at what time there are congestion on the path taken.

By using Google Direction which Google Direction itself is a web service to calculate the direction between locations. With this web service, users can get data travel time from start location to destination location. How to get the data travel time is with the input of the URL along with mandatory parameters and some optional parameters. The mandatory parameter entered into the URL is the origin in the form of a longitude and latitude point from the place of origin of departure, destination which is a point of longitude and latitude of the destination, and keys obtained from Google Console. The user embeds these parameters into the URL and will generate an output using a format. One of those formats is the JSON format.

A simple application to be built aims to extract data travel time from input request in one day for one week. The application is based on Java by using jsoup library to be able to request to service Google Direction and JSON library to extraction time travel data. Testing of this simple application is done by using a test case by making a request on one day. Based on the test results, the application can run well and provide output .csv files to be analyzed to provide the best time to travel with the help of Microsoft Excel applications. The results of testing this simple application proves that Google Direction API can be utilized to analyze the travel time between points in order to get the optimal travel time.} 
%=============================================================================

%_____________________________________________________________________________
%=============================================================================
% 								BAGIAN XI
%=============================================================================
% Kata-kata kunci dan keywords : diletakkan di bawah abstrak (ina dan eng)
% - kunciINA: kata-kata kunci dalam bahasa indonesia
% - kunciENG: keywords in english
%=============================================================================
\kunciINA{Kemacetan, Kota Bandung, Google Direction, JSON, Java, jsoup, Microsoft Excel.}
\kunciENG{Congestion, Bandung City, Google Direction, JSON, Java, jsoup, Microsoft Excel.}
%=============================================================================

%_____________________________________________________________________________
%=============================================================================
% 								BAGIAN XII
%=============================================================================
% Persembahan : kepada siapa anda mempersembahkan skripsi ini ...
%=============================================================================
\untuk{Ibunda dan diri sendiri}
%=============================================================================

%_____________________________________________________________________________
%=============================================================================
% 								BAGIAN XIII
%=============================================================================
% Kata Pengantar: tempat anda menuliskan kata pengantar dan ucapan terima 
% kasih kepada yang telah membantu anda bla bla bla ....  
%=============================================================================
\prakata{Puji syukur kepada Tuhan Yang Maha Esa atas seluruh berkat yang diberikan kepada penulis sehingga dapat menyelesaikan tugas akhir dengan judul \textbf{Analisis Waktu Tempuh Kota Bandung (Studi Kasus : Antara Universitas Katolik Parahyangan dan Amaya Residence; Antara Universitas Katolik Parahyangan dan Jalan Puspa Utara)} dengan baik. Penulis juga berterimakasih kepada pihak-pihak yang telah memberikan dukungan dan bantuan kepada penulis dalam menyelesaikan tugas akhir ini, yaitu :
\begin{enumerate}
	\item Ibunda, kakak dan kakak ipar yang selalu memberi dukungan kepada penulis.
	\item Bapak Pascal Alfadian sebagai dosen pembimbing yang telah membimbing penulis hingga dapat menyelesaikan tugas akhir ini.
	\item Bapak Chandra Wijaya dan Bapak Husnul Hakim sebagai dosen penguji yang telah membantu menguji tugas akhir ini.
	\item Ibu Mariskha Tri Adithia, Fernando B. L. Waang dan Frida Ayu Ananditya yang telah membantu penulis dalam mengembangkan diri.
	\item Dwi Pinta Larrasaty Permana yang selalu memberi dukungan kepada penulis secara moril.
	\item Pihak-pihak lain yang belum disebutkan, yang berperan dalah penyelesaian tugas akhir ini.
\end{enumerate}
Penulis menyadari bahwa tugas akhir ini masih jauh dari kesempurnaan, maka saran dan kritik yang konstruktif dari semua pihak diharapkan demi penyempurnaan selanjutnya.
Akhir kata, penulis berharap agar tugas akhir ini dapat bermanfaat bagi pembaca yang hendak melakukan penelitian dan pengembangan yang terkait dengan tugas akhir ini.
}
%=============================================================================

%_____________________________________________________________________________
%=============================================================================
% 								BAGIAN XIV
%=============================================================================
% Tambahkan hyphen (pemenggalan kata) yang anda butuhkan di sini 
%=============================================================================
\hyphenation{ma-te-ma-ti-ka}
\hyphenation{fi-si-ka}
\hyphenation{tek-nik}
\hyphenation{in-for-ma-ti-ka}
%=============================================================================

%_____________________________________________________________________________
%=============================================================================
% 								BAGIAN XV
%=============================================================================
% Tambahkan perintah yang anda buat sendiri di sini 
%=============================================================================
\newcommand{\vtemplateauthor}{lionov}
\pgfplotsset{compat=newest}
\usetikzlibrary{patterns}
%=============================================================================

% Copyright \textcopyright [Lionov] [09-10-2016]. All rights reserved